\documentclass[9pt]{beamer}

\mode<presentation> {\usetheme{Kaiping}}

\usepackage[ngerman]{babel}

\usepackage{fontspec}
\setmainfont{Linux Biolinum}
\setmonofont{Linux Biolinum}
\setsansfont{Linux Biolinum}

\usepackage{multimedia}
\usepackage{media9}

\usepackage{xcolor}
\usepackage{graphicx}
\usepackage{tikz-cd}

\usefonttheme[onlymath]{serif}
\usetheme{Warsaw}
% default
% Boadilla
% Madrid
% Pittsburgh
% Rochester \usetheme[height=7mm]{Rochester}
% Copenhagen
% Warsaw
% Singapore
% Malmoe

% Decrease footnote size

\title{Growing Trees from Big Data}
\subtitle{Bayesian Phyologeny for Historical Linguistics}
\author{Gereon Kaiping}
%\institute{Leiden University Centre for Linguistics, the Netherlands}
\date{20-3-2017}

\begin{document}

\begin{frame}[plain]
  \titlepage
\end{frame}
\begin{frame}
  \tableofcontents
\end{frame}
\section{The Problem}
\begin{frame}{Problem}
  Using the comparative method is hard, because
  \begin{itemize}
  \item it is a lot of painstaking work,
  \item we don't know how to weigh the evidence,
  \item cognates may have changed meanings,
  \item loan words make our lives more difficult.
  \end{itemize}
  And then doesn't even give us dates, just “not before” or “not after” if we are lucky.
\end{frame}
\section{The Solution!}
\begin{frame}{Solution}
\end{frame}
\subsection{Bayes' Theorem}
\begin{frame}{Bayes' Theorem}
\end{frame}
\subsection{Phylogenetics}
\begin{frame}{Computational Phylogenetics}
\end{frame}
\subsection{Examples}
\begin{frame}{Example 1:}
\end{frame}
\section{And why it is not.}
\begin{frame}{Criticism}
\end{frame}
\section{Conclusions}
\begin{frame}{Conclusions}
\end{frame}
\section{Further Reading}
\begin{frame}
  Course
  Literature: McMahon \& McMahon
\end{frame}
\end{document}
%%% Local Variables:
%%% TeX-engine: luatex
%%% End: 
